\documentclass[]{article}
\usepackage{lmodern}
\usepackage{amssymb,amsmath}
\usepackage{ifxetex,ifluatex}
\usepackage{fixltx2e} % provides \textsubscript
\ifnum 0\ifxetex 1\fi\ifluatex 1\fi=0 % if pdftex
  \usepackage[T1]{fontenc}
  \usepackage[utf8]{inputenc}
\else % if luatex or xelatex
  \ifxetex
    \usepackage{mathspec}
  \else
    \usepackage{fontspec}
  \fi
  \defaultfontfeatures{Ligatures=TeX,Scale=MatchLowercase}
\fi
% use upquote if available, for straight quotes in verbatim environments
\IfFileExists{upquote.sty}{\usepackage{upquote}}{}
% use microtype if available
\IfFileExists{microtype.sty}{%
\usepackage{microtype}
\UseMicrotypeSet[protrusion]{basicmath} % disable protrusion for tt fonts
}{}
\usepackage[margin=1in]{geometry}
\usepackage{hyperref}
\hypersetup{unicode=true,
            pdftitle={Analise de Series Temporais - Trabalho 1},
            pdfborder={0 0 0},
            breaklinks=true}
\urlstyle{same}  % don't use monospace font for urls
\usepackage{color}
\usepackage{fancyvrb}
\newcommand{\VerbBar}{|}
\newcommand{\VERB}{\Verb[commandchars=\\\{\}]}
\DefineVerbatimEnvironment{Highlighting}{Verbatim}{commandchars=\\\{\}}
% Add ',fontsize=\small' for more characters per line
\usepackage{framed}
\definecolor{shadecolor}{RGB}{248,248,248}
\newenvironment{Shaded}{\begin{snugshade}}{\end{snugshade}}
\newcommand{\KeywordTok}[1]{\textcolor[rgb]{0.13,0.29,0.53}{\textbf{#1}}}
\newcommand{\DataTypeTok}[1]{\textcolor[rgb]{0.13,0.29,0.53}{#1}}
\newcommand{\DecValTok}[1]{\textcolor[rgb]{0.00,0.00,0.81}{#1}}
\newcommand{\BaseNTok}[1]{\textcolor[rgb]{0.00,0.00,0.81}{#1}}
\newcommand{\FloatTok}[1]{\textcolor[rgb]{0.00,0.00,0.81}{#1}}
\newcommand{\ConstantTok}[1]{\textcolor[rgb]{0.00,0.00,0.00}{#1}}
\newcommand{\CharTok}[1]{\textcolor[rgb]{0.31,0.60,0.02}{#1}}
\newcommand{\SpecialCharTok}[1]{\textcolor[rgb]{0.00,0.00,0.00}{#1}}
\newcommand{\StringTok}[1]{\textcolor[rgb]{0.31,0.60,0.02}{#1}}
\newcommand{\VerbatimStringTok}[1]{\textcolor[rgb]{0.31,0.60,0.02}{#1}}
\newcommand{\SpecialStringTok}[1]{\textcolor[rgb]{0.31,0.60,0.02}{#1}}
\newcommand{\ImportTok}[1]{#1}
\newcommand{\CommentTok}[1]{\textcolor[rgb]{0.56,0.35,0.01}{\textit{#1}}}
\newcommand{\DocumentationTok}[1]{\textcolor[rgb]{0.56,0.35,0.01}{\textbf{\textit{#1}}}}
\newcommand{\AnnotationTok}[1]{\textcolor[rgb]{0.56,0.35,0.01}{\textbf{\textit{#1}}}}
\newcommand{\CommentVarTok}[1]{\textcolor[rgb]{0.56,0.35,0.01}{\textbf{\textit{#1}}}}
\newcommand{\OtherTok}[1]{\textcolor[rgb]{0.56,0.35,0.01}{#1}}
\newcommand{\FunctionTok}[1]{\textcolor[rgb]{0.00,0.00,0.00}{#1}}
\newcommand{\VariableTok}[1]{\textcolor[rgb]{0.00,0.00,0.00}{#1}}
\newcommand{\ControlFlowTok}[1]{\textcolor[rgb]{0.13,0.29,0.53}{\textbf{#1}}}
\newcommand{\OperatorTok}[1]{\textcolor[rgb]{0.81,0.36,0.00}{\textbf{#1}}}
\newcommand{\BuiltInTok}[1]{#1}
\newcommand{\ExtensionTok}[1]{#1}
\newcommand{\PreprocessorTok}[1]{\textcolor[rgb]{0.56,0.35,0.01}{\textit{#1}}}
\newcommand{\AttributeTok}[1]{\textcolor[rgb]{0.77,0.63,0.00}{#1}}
\newcommand{\RegionMarkerTok}[1]{#1}
\newcommand{\InformationTok}[1]{\textcolor[rgb]{0.56,0.35,0.01}{\textbf{\textit{#1}}}}
\newcommand{\WarningTok}[1]{\textcolor[rgb]{0.56,0.35,0.01}{\textbf{\textit{#1}}}}
\newcommand{\AlertTok}[1]{\textcolor[rgb]{0.94,0.16,0.16}{#1}}
\newcommand{\ErrorTok}[1]{\textcolor[rgb]{0.64,0.00,0.00}{\textbf{#1}}}
\newcommand{\NormalTok}[1]{#1}
\usepackage{graphicx,grffile}
\makeatletter
\def\maxwidth{\ifdim\Gin@nat@width>\linewidth\linewidth\else\Gin@nat@width\fi}
\def\maxheight{\ifdim\Gin@nat@height>\textheight\textheight\else\Gin@nat@height\fi}
\makeatother
% Scale images if necessary, so that they will not overflow the page
% margins by default, and it is still possible to overwrite the defaults
% using explicit options in \includegraphics[width, height, ...]{}
\setkeys{Gin}{width=\maxwidth,height=\maxheight,keepaspectratio}
\IfFileExists{parskip.sty}{%
\usepackage{parskip}
}{% else
\setlength{\parindent}{0pt}
\setlength{\parskip}{6pt plus 2pt minus 1pt}
}
\setlength{\emergencystretch}{3em}  % prevent overfull lines
\providecommand{\tightlist}{%
  \setlength{\itemsep}{0pt}\setlength{\parskip}{0pt}}
\setcounter{secnumdepth}{0}
% Redefines (sub)paragraphs to behave more like sections
\ifx\paragraph\undefined\else
\let\oldparagraph\paragraph
\renewcommand{\paragraph}[1]{\oldparagraph{#1}\mbox{}}
\fi
\ifx\subparagraph\undefined\else
\let\oldsubparagraph\subparagraph
\renewcommand{\subparagraph}[1]{\oldsubparagraph{#1}\mbox{}}
\fi

%%% Use protect on footnotes to avoid problems with footnotes in titles
\let\rmarkdownfootnote\footnote%
\def\footnote{\protect\rmarkdownfootnote}

%%% Change title format to be more compact
\usepackage{titling}

% Create subtitle command for use in maketitle
\newcommand{\subtitle}[1]{
  \posttitle{
    \begin{center}\large#1\end{center}
    }
}

\setlength{\droptitle}{-2em}
  \title{Analise de Series Temporais - Trabalho 1}
  \pretitle{\vspace{\droptitle}\centering\huge}
  \posttitle{\par}
  \author{}
  \preauthor{}\postauthor{}
  \date{}
  \predate{}\postdate{}


\begin{document}
\maketitle

Iniciando o ambiente

\begin{Shaded}
\begin{Highlighting}[]
\KeywordTok{cat}\NormalTok{(}\StringTok{"}\CharTok{\textbackslash{}014}\StringTok{"}\NormalTok{)  }
\end{Highlighting}
\end{Shaded}



\begin{Shaded}
\begin{Highlighting}[]
\KeywordTok{rm}\NormalTok{(}\DataTypeTok{list =} \KeywordTok{ls}\NormalTok{())}
\end{Highlighting}
\end{Shaded}

1- Utilizando o arquivo ``Serie\_Dados.csv'' realize as seguintes
etapas:

\begin{Shaded}
\begin{Highlighting}[]
\NormalTok{Serie_Dados <-}\StringTok{ }\KeywordTok{read.csv}\NormalTok{(}\StringTok{"Serie_Dados.csv"}\NormalTok{, }\DataTypeTok{sep=}\StringTok{";"}\NormalTok{)}
\CommentTok{#View(Serie_Dados)}
\end{Highlighting}
\end{Shaded}

\begin{enumerate}
\def\labelenumi{\alph{enumi})}
\tightlist
\item
  Crie a série temporal dos retornos Ln, ou seja, r=Ln(P\_t+1 /P\_t)
\end{enumerate}

\begin{Shaded}
\begin{Highlighting}[]
\NormalTok{Serie_Dados.LN <-}\StringTok{ }\KeywordTok{log}\NormalTok{(Serie_Dados[}\DecValTok{2}\OperatorTok{:}\DecValTok{13}\NormalTok{]}\OperatorTok{/}\KeywordTok{rbind}\NormalTok{(}\OtherTok{NA}\NormalTok{,Serie_Dados[}\DecValTok{2}\OperatorTok{:}\DecValTok{13}\NormalTok{][}\OperatorTok{-}\KeywordTok{nrow}\NormalTok{(Serie_Dados[}\DecValTok{2}\OperatorTok{:}\DecValTok{13}\NormalTok{]),]))}
\NormalTok{Serie_Dados.LN <-}\StringTok{ }\NormalTok{Serie_Dados.LN[}\OperatorTok{-}\DecValTok{1}\NormalTok{,]}
\end{Highlighting}
\end{Shaded}

\begin{enumerate}
\def\labelenumi{\alph{enumi})}
\setcounter{enumi}{1}
\tightlist
\item
  Para cada ação construa o histograma dos retornos. Comente o resultado
  dos histogramas, verifique também o desvio padrão e a média de cada
  série
\end{enumerate}

Histogramas:

\begin{Shaded}
\begin{Highlighting}[]
\ControlFlowTok{for}\NormalTok{ (col }\ControlFlowTok{in} \DecValTok{1}\OperatorTok{:}\KeywordTok{ncol}\NormalTok{(Serie_Dados.LN)) \{}
  \KeywordTok{hist}\NormalTok{(Serie_Dados.LN[,col], }\DataTypeTok{main =} \KeywordTok{names}\NormalTok{(Serie_Dados.LN[col]), }\DataTypeTok{xlab =} \StringTok{""}\NormalTok{)}
\NormalTok{\}}
\end{Highlighting}
\end{Shaded}

\includegraphics{Trabalho_I_files/figure-latex/unnamed-chunk-4-1.pdf}
\includegraphics{Trabalho_I_files/figure-latex/unnamed-chunk-4-2.pdf}
\includegraphics{Trabalho_I_files/figure-latex/unnamed-chunk-4-3.pdf}
\includegraphics{Trabalho_I_files/figure-latex/unnamed-chunk-4-4.pdf}
\includegraphics{Trabalho_I_files/figure-latex/unnamed-chunk-4-5.pdf}
\includegraphics{Trabalho_I_files/figure-latex/unnamed-chunk-4-6.pdf}
\includegraphics{Trabalho_I_files/figure-latex/unnamed-chunk-4-7.pdf}
\includegraphics{Trabalho_I_files/figure-latex/unnamed-chunk-4-8.pdf}
\includegraphics{Trabalho_I_files/figure-latex/unnamed-chunk-4-9.pdf}
\includegraphics{Trabalho_I_files/figure-latex/unnamed-chunk-4-10.pdf}
\includegraphics{Trabalho_I_files/figure-latex/unnamed-chunk-4-11.pdf}
\includegraphics{Trabalho_I_files/figure-latex/unnamed-chunk-4-12.pdf}
Desvio Padrao e Media:

\begin{Shaded}
\begin{Highlighting}[]
\KeywordTok{sapply}\NormalTok{(Serie_Dados.LN, }\ControlFlowTok{function}\NormalTok{(cl) }\KeywordTok{list}\NormalTok{(}\DataTypeTok{Media=}\KeywordTok{mean}\NormalTok{(cl,}\DataTypeTok{na.rm=}\OtherTok{TRUE}\NormalTok{), }\DataTypeTok{DesvioPadrao=}\KeywordTok{sd}\NormalTok{(cl,}\DataTypeTok{na.rm=}\OtherTok{TRUE}\NormalTok{)))}
\end{Highlighting}
\end{Shaded}

\begin{verbatim}
##              VALE5        GOLL4        AMBV4       ITUB4        
## Media        0.0001293889 -0.000491604 0.001271904 -3.197417e-05
## DesvioPadrao 0.01838984   0.0324803    0.01426466  0.01833354   
##              BBDC4       BVMF3        RAPT4        MYPK3      
## Media        0.000209532 9.431336e-05 0.0003498633 0.001024337
## DesvioPadrao 0.01708826  0.02193809   0.02011623   0.02255594 
##              GOAU4         LLXL3        CSAN3        DOLAR       
## Media        -0.0001969214 -0.001223711 0.0008827886 0.0002516147
## DesvioPadrao 0.02253222    0.04117323   0.01928199   0.007719852
\end{verbatim}

\begin{enumerate}
\def\labelenumi{\alph{enumi})}
\setcounter{enumi}{2}
\tightlist
\item
  Calcule o ACF e PACF de cada série de retornos. Comente os resultados.
\end{enumerate}

ACF

\begin{Shaded}
\begin{Highlighting}[]
\ControlFlowTok{for}\NormalTok{ (col }\ControlFlowTok{in} \DecValTok{1}\OperatorTok{:}\KeywordTok{ncol}\NormalTok{(Serie_Dados.LN)) \{}
  \KeywordTok{acf}\NormalTok{(Serie_Dados.LN[,col], }\DataTypeTok{main =} \KeywordTok{names}\NormalTok{(Serie_Dados.LN[col]), }\DataTypeTok{xlab =} \StringTok{""}\NormalTok{)}
\NormalTok{\}}
\end{Highlighting}
\end{Shaded}

\includegraphics{Trabalho_I_files/figure-latex/unnamed-chunk-6-1.pdf}
\includegraphics{Trabalho_I_files/figure-latex/unnamed-chunk-6-2.pdf}
\includegraphics{Trabalho_I_files/figure-latex/unnamed-chunk-6-3.pdf}
\includegraphics{Trabalho_I_files/figure-latex/unnamed-chunk-6-4.pdf}
\includegraphics{Trabalho_I_files/figure-latex/unnamed-chunk-6-5.pdf}
\includegraphics{Trabalho_I_files/figure-latex/unnamed-chunk-6-6.pdf}
\includegraphics{Trabalho_I_files/figure-latex/unnamed-chunk-6-7.pdf}
\includegraphics{Trabalho_I_files/figure-latex/unnamed-chunk-6-8.pdf}
\includegraphics{Trabalho_I_files/figure-latex/unnamed-chunk-6-9.pdf}
\includegraphics{Trabalho_I_files/figure-latex/unnamed-chunk-6-10.pdf}
\includegraphics{Trabalho_I_files/figure-latex/unnamed-chunk-6-11.pdf}
\includegraphics{Trabalho_I_files/figure-latex/unnamed-chunk-6-12.pdf}

PACF

\begin{Shaded}
\begin{Highlighting}[]
\ControlFlowTok{for}\NormalTok{ (col }\ControlFlowTok{in} \DecValTok{1}\OperatorTok{:}\KeywordTok{ncol}\NormalTok{(Serie_Dados.LN)) \{}
  \KeywordTok{pacf}\NormalTok{(Serie_Dados.LN[,col], }\DataTypeTok{main =} \KeywordTok{names}\NormalTok{(Serie_Dados.LN[col]), }\DataTypeTok{xlab =} \StringTok{""}\NormalTok{)}
\NormalTok{\}}
\end{Highlighting}
\end{Shaded}

\includegraphics{Trabalho_I_files/figure-latex/unnamed-chunk-7-1.pdf}
\includegraphics{Trabalho_I_files/figure-latex/unnamed-chunk-7-2.pdf}
\includegraphics{Trabalho_I_files/figure-latex/unnamed-chunk-7-3.pdf}
\includegraphics{Trabalho_I_files/figure-latex/unnamed-chunk-7-4.pdf}
\includegraphics{Trabalho_I_files/figure-latex/unnamed-chunk-7-5.pdf}
\includegraphics{Trabalho_I_files/figure-latex/unnamed-chunk-7-6.pdf}
\includegraphics{Trabalho_I_files/figure-latex/unnamed-chunk-7-7.pdf}
\includegraphics{Trabalho_I_files/figure-latex/unnamed-chunk-7-8.pdf}
\includegraphics{Trabalho_I_files/figure-latex/unnamed-chunk-7-9.pdf}
\includegraphics{Trabalho_I_files/figure-latex/unnamed-chunk-7-10.pdf}
\includegraphics{Trabalho_I_files/figure-latex/unnamed-chunk-7-11.pdf}
\includegraphics{Trabalho_I_files/figure-latex/unnamed-chunk-7-12.pdf}

2- Para cada um dos processos abaixo gere 200 observações. Faça um
gráfico da série, ACF e PACF. Comente os resultados.

Definindo uma semente para os numeros aleatorios serem sempre os mesmos:

\begin{Shaded}
\begin{Highlighting}[]
\KeywordTok{set.seed}\NormalTok{(}\DecValTok{1234}\NormalTok{)}
\end{Highlighting}
\end{Shaded}

\begin{enumerate}
\def\labelenumi{\alph{enumi})}
\setcounter{enumi}{3}
\tightlist
\item
  Série aleatória, observações iid da distribuição N(0,1)
\end{enumerate}

\begin{Shaded}
\begin{Highlighting}[]
\CommentTok{#x.iid = data.frame(t = x[2:200],}
\CommentTok{#                 t_1 = x[1:199])}
\CommentTok{#x.iid}
\CommentTok{#x.iid.mod = lm(t~t_1,data = x.iid)}
\CommentTok{#summary(x.iid.mod)}
\CommentTok{#plot(x.iid.mod)}
\CommentTok{#acf(x.iid)}
\CommentTok{#pacf(x.iid)}

\KeywordTok{par}\NormalTok{(}\DataTypeTok{mfrow =} \KeywordTok{c}\NormalTok{(}\DecValTok{1}\NormalTok{, }\DecValTok{3}\NormalTok{))}

\NormalTok{d <-}\StringTok{ }\KeywordTok{ts}\NormalTok{(}\KeywordTok{rnorm}\NormalTok{(}\DecValTok{200}\NormalTok{, }\DecValTok{0}\NormalTok{, }\DecValTok{1}\NormalTok{)) }\CommentTok{#ts: time series}
\KeywordTok{plot}\NormalTok{(d)}
\KeywordTok{acf}\NormalTok{(d)}
\KeywordTok{pacf}\NormalTok{(d)}
\end{Highlighting}
\end{Shaded}

\includegraphics{Trabalho_I_files/figure-latex/unnamed-chunk-9-1.pdf}

A serie é estacionária, mas por ser iid tem a PACF igual a 0

\begin{enumerate}
\def\labelenumi{\alph{enumi})}
\setcounter{enumi}{4}
\tightlist
\item
  Série com tendência estocástica,
\end{enumerate}

Neste caso o coeficiente tem que ser menor que 1 para rodar, então ar =
0.99999

\begin{Shaded}
\begin{Highlighting}[]
\NormalTok{e <-}\StringTok{ }\KeywordTok{arima.sim}\NormalTok{(}\DataTypeTok{model =} \KeywordTok{list}\NormalTok{(}\DataTypeTok{ar=} \FloatTok{0.99999}\NormalTok{), }\DataTypeTok{n=}\DecValTok{200}\NormalTok{, }\DataTypeTok{innov =} \KeywordTok{rnorm}\NormalTok{(}\DecValTok{200}\NormalTok{,}\DecValTok{1}\NormalTok{,}\DecValTok{25}\NormalTok{))}
\KeywordTok{plot}\NormalTok{(e)}
\end{Highlighting}
\end{Shaded}

\includegraphics{Trabalho_I_files/figure-latex/unnamed-chunk-10-1.pdf}

\begin{Shaded}
\begin{Highlighting}[]
\KeywordTok{acf}\NormalTok{(e)}
\end{Highlighting}
\end{Shaded}

\includegraphics{Trabalho_I_files/figure-latex/unnamed-chunk-10-2.pdf}

\begin{Shaded}
\begin{Highlighting}[]
\KeywordTok{pacf}\NormalTok{(e)}
\end{Highlighting}
\end{Shaded}

\includegraphics{Trabalho_I_files/figure-latex/unnamed-chunk-10-3.pdf}

A série não é estacionaria

\begin{enumerate}
\def\labelenumi{\alph{enumi})}
\setcounter{enumi}{5}
\tightlist
\item
  Serie com correlação de curto-prazo,
\end{enumerate}

\begin{Shaded}
\begin{Highlighting}[]
\NormalTok{f <-}\StringTok{ }\KeywordTok{arima.sim}\NormalTok{(}\DataTypeTok{model =} \KeywordTok{list}\NormalTok{(}\DataTypeTok{ar =} \FloatTok{0.7}\NormalTok{), }\DataTypeTok{n =} \DecValTok{200}\NormalTok{, }\DataTypeTok{innov =} \KeywordTok{rnorm}\NormalTok{(}\DecValTok{200}\NormalTok{, }\DecValTok{0}\NormalTok{, }\DecValTok{1}\NormalTok{))}
\KeywordTok{plot}\NormalTok{(f)}
\end{Highlighting}
\end{Shaded}

\includegraphics{Trabalho_I_files/figure-latex/unnamed-chunk-11-1.pdf}

\begin{Shaded}
\begin{Highlighting}[]
\KeywordTok{acf}\NormalTok{(f)}
\end{Highlighting}
\end{Shaded}

\includegraphics{Trabalho_I_files/figure-latex/unnamed-chunk-11-2.pdf}

\begin{Shaded}
\begin{Highlighting}[]
\KeywordTok{pacf}\NormalTok{(f)}
\end{Highlighting}
\end{Shaded}

\includegraphics{Trabalho_I_files/figure-latex/unnamed-chunk-11-3.pdf}

A série é estacionária, ACF com decaimento e grafico de pacf com pico em
1

\begin{enumerate}
\def\labelenumi{\alph{enumi})}
\setcounter{enumi}{6}
\tightlist
\item
  Série com correlações negativas
\end{enumerate}

\begin{Shaded}
\begin{Highlighting}[]
\NormalTok{g <-}\StringTok{ }\KeywordTok{arima.sim}\NormalTok{(}\DataTypeTok{model =} \KeywordTok{list}\NormalTok{(}\DataTypeTok{ar =} \OperatorTok{-}\FloatTok{0.7}\NormalTok{), }\DataTypeTok{n =} \DecValTok{200}\NormalTok{, }\DataTypeTok{innov =} \KeywordTok{rnorm}\NormalTok{(}\DecValTok{200}\NormalTok{, }\DecValTok{0}\NormalTok{, }\DecValTok{1}\NormalTok{))}
\KeywordTok{plot}\NormalTok{(g)}
\end{Highlighting}
\end{Shaded}

\includegraphics{Trabalho_I_files/figure-latex/unnamed-chunk-12-1.pdf}

\begin{Shaded}
\begin{Highlighting}[]
\KeywordTok{acf}\NormalTok{(g)}
\end{Highlighting}
\end{Shaded}

\includegraphics{Trabalho_I_files/figure-latex/unnamed-chunk-12-2.pdf}

\begin{Shaded}
\begin{Highlighting}[]
\KeywordTok{pacf}\NormalTok{(g)}
\end{Highlighting}
\end{Shaded}

\includegraphics{Trabalho_I_files/figure-latex/unnamed-chunk-12-3.pdf}

A série é estacionária, acf com decaimento oscilando, pacf pico em 1

\begin{enumerate}
\def\labelenumi{\alph{enumi})}
\setcounter{enumi}{7}
\tightlist
\item
  Medias moveis,
\end{enumerate}

\begin{Shaded}
\begin{Highlighting}[]
\NormalTok{h <-}\StringTok{ }\KeywordTok{arima.sim}\NormalTok{(}\DataTypeTok{model =} \KeywordTok{list}\NormalTok{(}\DataTypeTok{ar =} \FloatTok{0.6}\NormalTok{), }\DataTypeTok{n =} \DecValTok{200}\NormalTok{, }\DataTypeTok{innov =} \KeywordTok{rnorm}\NormalTok{(}\DecValTok{200}\NormalTok{, }\DecValTok{0}\NormalTok{, }\DecValTok{1}\NormalTok{))}
\KeywordTok{plot}\NormalTok{(h)}
\end{Highlighting}
\end{Shaded}

\includegraphics{Trabalho_I_files/figure-latex/unnamed-chunk-13-1.pdf}

\begin{Shaded}
\begin{Highlighting}[]
\KeywordTok{acf}\NormalTok{(h)}
\end{Highlighting}
\end{Shaded}

\includegraphics{Trabalho_I_files/figure-latex/unnamed-chunk-13-2.pdf}

\begin{Shaded}
\begin{Highlighting}[]
\KeywordTok{pacf}\NormalTok{(h)}
\end{Highlighting}
\end{Shaded}

\includegraphics{Trabalho_I_files/figure-latex/unnamed-chunk-13-3.pdf}

A série é estacionaria, acf igual a 0 em K\textgreater{}1 e pacf com
decaimento oscilando

3- Utilize a série abaixo para resolver cada item.

An example of a time series that can probably be described using an
additive model with a trend and no seasonality is the time series of the
annual diameter of women's skirts at the hem, from 1866 to 1911. The
data is available in the file
\url{http://robjhyndman.com/tsdldata/roberts/skirts.dat} (original data
from Hipel and McLeod, 1994).

\begin{Shaded}
\begin{Highlighting}[]
\NormalTok{skirts <-}\StringTok{ }\KeywordTok{read.table}\NormalTok{(}\StringTok{"http://robjhyndman.com/tsdldata/roberts/skirts.dat"}\NormalTok{, }\DataTypeTok{header =} \OtherTok{TRUE}\NormalTok{, }\DataTypeTok{skip =} \DecValTok{3}\NormalTok{) }
\end{Highlighting}
\end{Shaded}

\begin{enumerate}
\def\labelenumi{\alph{enumi})}
\tightlist
\item
  Faça a leitura da série de dados e os tratamentos necessários para
  considerar a mesma como uma série temporal
\end{enumerate}

\begin{Shaded}
\begin{Highlighting}[]
\NormalTok{skirts.ts<-}\KeywordTok{ts}\NormalTok{(skirts, }\DataTypeTok{frequency=}\DecValTok{1}\NormalTok{, }\DataTypeTok{start=}\KeywordTok{c}\NormalTok{(}\DecValTok{1866}\NormalTok{))}
\NormalTok{skirts.ts}
\end{Highlighting}
\end{Shaded}

\begin{verbatim}
## Time Series:
## Start = 1866 
## End = 1911 
## Frequency = 1 
##       SKIRTS
##  [1,]    608
##  [2,]    617
##  [3,]    625
##  [4,]    636
##  [5,]    657
##  [6,]    691
##  [7,]    728
##  [8,]    784
##  [9,]    816
## [10,]    876
## [11,]    949
## [12,]    997
## [13,]   1027
## [14,]   1047
## [15,]   1049
## [16,]   1018
## [17,]   1021
## [18,]   1012
## [19,]   1018
## [20,]    991
## [21,]    962
## [22,]    921
## [23,]    871
## [24,]    829
## [25,]    822
## [26,]    820
## [27,]    802
## [28,]    821
## [29,]    819
## [30,]    791
## [31,]    746
## [32,]    726
## [33,]    661
## [34,]    620
## [35,]    588
## [36,]    568
## [37,]    542
## [38,]    551
## [39,]    541
## [40,]    557
## [41,]    556
## [42,]    534
## [43,]    528
## [44,]    529
## [45,]    523
## [46,]    531
\end{verbatim}

\begin{enumerate}
\def\labelenumi{\alph{enumi})}
\setcounter{enumi}{1}
\tightlist
\item
  Faça a decomposição da série do item (a): Sazonalidade, Tendência e
  Aleatória.
\end{enumerate}

\begin{Shaded}
\begin{Highlighting}[]
\NormalTok{skirts.components <-}\StringTok{ }\KeywordTok{ifelse}\NormalTok{(}\KeywordTok{frequency}\NormalTok{(skirts.ts)}\OperatorTok{>}\DecValTok{1}\NormalTok{,}
                        \KeywordTok{decompose}\NormalTok{(skirts.ts,}\DataTypeTok{type =} \KeywordTok{c}\NormalTok{(}\StringTok{"additive"}\NormalTok{, }\StringTok{"multiplicative"}\NormalTok{)),}
                        \KeywordTok{print}\NormalTok{(}\StringTok{"Nao e' possivel decompor uma serie anual, para ser feita a decomposicao a serie deveria ter, no minimo, 2 periodos"}\NormalTok{))}
\end{Highlighting}
\end{Shaded}

\begin{verbatim}
## [1] "Nao e' possivel decompor uma serie anual, para ser feita a decomposicao a serie deveria ter, no minimo, 2 periodos"
\end{verbatim}

\begin{Shaded}
\begin{Highlighting}[]
\CommentTok{#plot(skirts.components)}
\end{Highlighting}
\end{Shaded}

\begin{enumerate}
\def\labelenumi{\alph{enumi})}
\setcounter{enumi}{2}
\tightlist
\item
  Calcule a ACF e PACF da série e da primeira diferença
\end{enumerate}

\begin{Shaded}
\begin{Highlighting}[]
\KeywordTok{acf}\NormalTok{(skirts.ts)}
\end{Highlighting}
\end{Shaded}

\includegraphics{Trabalho_I_files/figure-latex/unnamed-chunk-17-1.pdf}

\begin{Shaded}
\begin{Highlighting}[]
\KeywordTok{acf}\NormalTok{(}\KeywordTok{diff}\NormalTok{(skirts.ts))}
\end{Highlighting}
\end{Shaded}

\includegraphics{Trabalho_I_files/figure-latex/unnamed-chunk-17-2.pdf}

\begin{Shaded}
\begin{Highlighting}[]
\KeywordTok{pacf}\NormalTok{(skirts.ts)}
\end{Highlighting}
\end{Shaded}

\includegraphics{Trabalho_I_files/figure-latex/unnamed-chunk-17-3.pdf}

\begin{Shaded}
\begin{Highlighting}[]
\KeywordTok{pacf}\NormalTok{(}\KeywordTok{diff}\NormalTok{(skirts.ts))}
\end{Highlighting}
\end{Shaded}

\includegraphics{Trabalho_I_files/figure-latex/unnamed-chunk-17-4.pdf}

4- Usando a função arima.sim gere as seguintes simulações (300 ptos):

\begin{Shaded}
\begin{Highlighting}[]
\KeywordTok{set.seed}\NormalTok{(}\DecValTok{1234}\NormalTok{)}
\end{Highlighting}
\end{Shaded}

\begin{enumerate}
\def\labelenumi{\alph{enumi})}
\tightlist
\item
  Processo AR(1) onde θ0=0, θ1=0.7
\end{enumerate}

\begin{Shaded}
\begin{Highlighting}[]
\NormalTok{a <-}\StringTok{ }\KeywordTok{arima.sim}\NormalTok{(}\DataTypeTok{n=}\DecValTok{300}\NormalTok{,}\KeywordTok{list}\NormalTok{(}\DataTypeTok{ar =} \KeywordTok{c}\NormalTok{(}\DecValTok{0}\NormalTok{,.}\DecValTok{7}\NormalTok{)))}
\KeywordTok{plot}\NormalTok{(a)}
\end{Highlighting}
\end{Shaded}

\includegraphics{Trabalho_I_files/figure-latex/unnamed-chunk-19-1.pdf}

\begin{Shaded}
\begin{Highlighting}[]
\KeywordTok{acf}\NormalTok{(a)}
\end{Highlighting}
\end{Shaded}

\includegraphics{Trabalho_I_files/figure-latex/unnamed-chunk-19-2.pdf}

\begin{Shaded}
\begin{Highlighting}[]
\KeywordTok{pacf}\NormalTok{(a)}
\end{Highlighting}
\end{Shaded}

\includegraphics{Trabalho_I_files/figure-latex/unnamed-chunk-19-3.pdf}

\begin{enumerate}
\def\labelenumi{\alph{enumi})}
\setcounter{enumi}{1}
\tightlist
\item
  Processo AR(1) onde θ0=0, θ1= -0.7
\end{enumerate}

\begin{Shaded}
\begin{Highlighting}[]
\NormalTok{b <-}\StringTok{ }\KeywordTok{arima.sim}\NormalTok{(}\DataTypeTok{n=}\DecValTok{300}\NormalTok{,}\KeywordTok{list}\NormalTok{(}\DataTypeTok{ar =} \KeywordTok{c}\NormalTok{(}\DecValTok{0}\NormalTok{,}\OperatorTok{-}\NormalTok{.}\DecValTok{7}\NormalTok{)))}
\KeywordTok{plot}\NormalTok{(b)}
\end{Highlighting}
\end{Shaded}

\includegraphics{Trabalho_I_files/figure-latex/unnamed-chunk-20-1.pdf}

\begin{Shaded}
\begin{Highlighting}[]
\KeywordTok{acf}\NormalTok{(b)}
\end{Highlighting}
\end{Shaded}

\includegraphics{Trabalho_I_files/figure-latex/unnamed-chunk-20-2.pdf}

\begin{Shaded}
\begin{Highlighting}[]
\KeywordTok{pacf}\NormalTok{(b)}
\end{Highlighting}
\end{Shaded}

\includegraphics{Trabalho_I_files/figure-latex/unnamed-chunk-20-3.pdf}

\begin{enumerate}
\def\labelenumi{\alph{enumi})}
\setcounter{enumi}{2}
\tightlist
\item
  Processo AR(2) onde θ0=0, θ1=0.3 e θ2=0.5
\end{enumerate}

\begin{Shaded}
\begin{Highlighting}[]
\NormalTok{c <-}\StringTok{ }\KeywordTok{arima.sim}\NormalTok{(}\DataTypeTok{n=}\DecValTok{300}\NormalTok{,}\KeywordTok{list}\NormalTok{(}\DataTypeTok{ar =} \KeywordTok{c}\NormalTok{(}\DecValTok{0}\NormalTok{,.}\DecValTok{3}\NormalTok{,.}\DecValTok{5}\NormalTok{)))}
\KeywordTok{plot}\NormalTok{(c)}
\end{Highlighting}
\end{Shaded}

\includegraphics{Trabalho_I_files/figure-latex/unnamed-chunk-21-1.pdf}

\begin{Shaded}
\begin{Highlighting}[]
\KeywordTok{acf}\NormalTok{(c)}
\end{Highlighting}
\end{Shaded}

\includegraphics{Trabalho_I_files/figure-latex/unnamed-chunk-21-2.pdf}

\begin{Shaded}
\begin{Highlighting}[]
\KeywordTok{pacf}\NormalTok{(c)}
\end{Highlighting}
\end{Shaded}

\includegraphics{Trabalho_I_files/figure-latex/unnamed-chunk-21-3.pdf}

\begin{enumerate}
\def\labelenumi{\alph{enumi})}
\setcounter{enumi}{3}
\tightlist
\item
  Processo MA(1) onde θ0=0, θ1=0.6
\end{enumerate}

\begin{Shaded}
\begin{Highlighting}[]
\NormalTok{d <-}\StringTok{ }\KeywordTok{arima.sim}\NormalTok{(}\DataTypeTok{n=}\DecValTok{300}\NormalTok{,}\KeywordTok{list}\NormalTok{(}\DataTypeTok{ma =} \KeywordTok{c}\NormalTok{(}\DecValTok{0}\NormalTok{,.}\DecValTok{6}\NormalTok{)))}
\KeywordTok{plot}\NormalTok{(d)}
\end{Highlighting}
\end{Shaded}

\includegraphics{Trabalho_I_files/figure-latex/unnamed-chunk-22-1.pdf}

\begin{Shaded}
\begin{Highlighting}[]
\KeywordTok{acf}\NormalTok{(d)}
\end{Highlighting}
\end{Shaded}

\includegraphics{Trabalho_I_files/figure-latex/unnamed-chunk-22-2.pdf}

\begin{Shaded}
\begin{Highlighting}[]
\KeywordTok{pacf}\NormalTok{(d)}
\end{Highlighting}
\end{Shaded}

\includegraphics{Trabalho_I_files/figure-latex/unnamed-chunk-22-3.pdf}

\begin{enumerate}
\def\labelenumi{\alph{enumi})}
\setcounter{enumi}{4}
\tightlist
\item
  Processo MA(1) onde θ0=0, θ1=-0.6
\end{enumerate}

\begin{Shaded}
\begin{Highlighting}[]
\NormalTok{e <-}\StringTok{ }\KeywordTok{arima.sim}\NormalTok{(}\DataTypeTok{n=}\DecValTok{300}\NormalTok{,}\KeywordTok{list}\NormalTok{(}\DataTypeTok{ma =} \KeywordTok{c}\NormalTok{(}\DecValTok{0}\NormalTok{,}\OperatorTok{-}\NormalTok{.}\DecValTok{6}\NormalTok{)))}
\KeywordTok{plot}\NormalTok{(e)}
\end{Highlighting}
\end{Shaded}

\includegraphics{Trabalho_I_files/figure-latex/unnamed-chunk-23-1.pdf}

\begin{Shaded}
\begin{Highlighting}[]
\KeywordTok{acf}\NormalTok{(e)}
\end{Highlighting}
\end{Shaded}

\includegraphics{Trabalho_I_files/figure-latex/unnamed-chunk-23-2.pdf}

\begin{Shaded}
\begin{Highlighting}[]
\KeywordTok{pacf}\NormalTok{(e)}
\end{Highlighting}
\end{Shaded}

\includegraphics{Trabalho_I_files/figure-latex/unnamed-chunk-23-3.pdf}

Para cada simulação, plote o gráfico da série, calcule o ACF e PACF.
Usando estes resultados conclua como deve ser o comportamento da ACF de
PACF de um modelo autoregressivo( AR.)

5- Obtenha a série histórica do PIB Brasil no site:
\url{http://www.bcb.gov.br/pre/portalCidadao/cadsis/series.asp?idpai=PORTALBCB}
Código da série: 1232

\begin{Shaded}
\begin{Highlighting}[]
\NormalTok{PIB <-}\StringTok{ }\KeywordTok{read.csv}\NormalTok{(}\StringTok{"PIB.csv"}\NormalTok{, }\DataTypeTok{sep=}\StringTok{";"}\NormalTok{)}
\CommentTok{#View(PIB)}
\NormalTok{PIB.ts<-}\KeywordTok{ts}\NormalTok{(PIB}\OperatorTok{$}\NormalTok{X1232.PIB, }\DataTypeTok{frequency=}\DecValTok{4}\NormalTok{, }\DataTypeTok{start=}\KeywordTok{c}\NormalTok{(}\DecValTok{1991}\NormalTok{,}\DecValTok{1}\NormalTok{))}
\end{Highlighting}
\end{Shaded}

\begin{enumerate}
\def\labelenumi{\alph{enumi})}
\tightlist
\item
  Plote o gráfico da série usando o R
\end{enumerate}

\begin{Shaded}
\begin{Highlighting}[]
\KeywordTok{plot.ts}\NormalTok{(PIB.ts)}
\end{Highlighting}
\end{Shaded}

\includegraphics{Trabalho_I_files/figure-latex/unnamed-chunk-25-1.pdf}

\begin{enumerate}
\def\labelenumi{\alph{enumi})}
\setcounter{enumi}{1}
\tightlist
\item
  Faça a decomposição da série em: Sazonalidade, Tendência e Aleatória.
\end{enumerate}

\begin{Shaded}
\begin{Highlighting}[]
\NormalTok{PIB.decomposto <-}\StringTok{ }\KeywordTok{decompose}\NormalTok{(PIB.ts)}
\NormalTok{PIB.decomposto}
\end{Highlighting}
\end{Shaded}

\begin{verbatim}
## $x
##      Qtr1 Qtr2 Qtr3 Qtr4
## 1991   75   80   84   78
## 1992   74   77   82   83
## 1993   76   81   87   86
## 1994   79   85   92    5
## 1995   89   91    3    4
## 1996   88    2   11    6
## 1997   90    8   18   12
## 1998   93   10   17    8
## 1999    1    9   14   13
## 2000    7   19   22   23
## 2001   15   25   24   21
## 2002   16   26   29   30
## 2003   20   28   31   32
## 2004   27   34   35   36
## 2005   33   40   38   39
## 2006   37   41   43   44
## 2007   42   46   48   49
## 2008   47   52   55   50
## 2009   45   51   53   56
## 2010   54   59   60   61
## 2011   57   63   65   62
## 2012   58   66   67   68
## 2013   61   73   71   72
## 2014   64   69   70   94
## 
## $seasonal
##            Qtr1       Qtr2       Qtr3       Qtr4
## 1991  5.2853261  0.1440217 -0.3614130 -5.0679348
## 1992  5.2853261  0.1440217 -0.3614130 -5.0679348
## 1993  5.2853261  0.1440217 -0.3614130 -5.0679348
## 1994  5.2853261  0.1440217 -0.3614130 -5.0679348
## 1995  5.2853261  0.1440217 -0.3614130 -5.0679348
## 1996  5.2853261  0.1440217 -0.3614130 -5.0679348
## 1997  5.2853261  0.1440217 -0.3614130 -5.0679348
## 1998  5.2853261  0.1440217 -0.3614130 -5.0679348
## 1999  5.2853261  0.1440217 -0.3614130 -5.0679348
## 2000  5.2853261  0.1440217 -0.3614130 -5.0679348
## 2001  5.2853261  0.1440217 -0.3614130 -5.0679348
## 2002  5.2853261  0.1440217 -0.3614130 -5.0679348
## 2003  5.2853261  0.1440217 -0.3614130 -5.0679348
## 2004  5.2853261  0.1440217 -0.3614130 -5.0679348
## 2005  5.2853261  0.1440217 -0.3614130 -5.0679348
## 2006  5.2853261  0.1440217 -0.3614130 -5.0679348
## 2007  5.2853261  0.1440217 -0.3614130 -5.0679348
## 2008  5.2853261  0.1440217 -0.3614130 -5.0679348
## 2009  5.2853261  0.1440217 -0.3614130 -5.0679348
## 2010  5.2853261  0.1440217 -0.3614130 -5.0679348
## 2011  5.2853261  0.1440217 -0.3614130 -5.0679348
## 2012  5.2853261  0.1440217 -0.3614130 -5.0679348
## 2013  5.2853261  0.1440217 -0.3614130 -5.0679348
## 2014  5.2853261  0.1440217 -0.3614130 -5.0679348
## 
## $trend
##        Qtr1   Qtr2   Qtr3   Qtr4
## 1991     NA     NA 79.125 78.625
## 1992 78.000 78.375 79.250 80.000
## 1993 81.125 82.125 82.875 83.750
## 1994 84.875 75.375 66.500 68.500
## 1995 58.125 46.875 46.625 35.375
## 1996 25.250 26.500 27.000 28.000
## 1997 29.625 31.250 32.375 33.000
## 1998 33.125 32.500 20.500  8.875
## 1999  8.375  8.625 10.000 12.000
## 2000 14.250 16.500 18.750 20.500
## 2001 21.500 21.500 21.375 21.625
## 2002 22.375 24.125 25.750 26.500
## 2003 27.000 27.500 28.625 30.250
## 2004 31.500 32.500 33.750 35.250
## 2005 36.375 37.125 38.000 38.625
## 2006 39.375 40.625 41.875 43.125
## 2007 44.375 45.625 46.875 48.250
## 2008 49.875 50.875 50.750 50.375
## 2009 50.000 50.500 52.375 54.500
## 2010 56.375 57.875 58.875 59.750
## 2011 60.875 61.625 61.875 62.375
## 2012 63.000 64.000 65.125 66.375
## 2013 67.750 68.750 69.625 69.500
## 2014 68.875 71.500     NA     NA
## 
## $random
##             Qtr1        Qtr2        Qtr3        Qtr4
## 1991          NA          NA   5.2364130   4.4429348
## 1992  -9.2853261  -1.5190217   3.1114130   8.0679348
## 1993 -10.4103261  -1.2690217   4.4864130   7.3179348
## 1994 -11.1603261   9.4809783  25.8614130 -58.4320652
## 1995  25.5896739  43.9809783 -43.2635870 -26.3070652
## 1996  57.4646739 -24.6440217 -15.6385870 -16.9320652
## 1997  55.0896739 -23.3940217 -14.0135870 -15.9320652
## 1998  54.5896739 -22.6440217  -3.1385870   4.1929348
## 1999 -12.6603261   0.2309783   4.3614130   6.0679348
## 2000 -12.5353261   2.3559783   3.6114130   7.5679348
## 2001 -11.7853261   3.3559783   2.9864130   4.4429348
## 2002 -11.6603261   1.7309783   3.6114130   8.5679348
## 2003 -12.2853261   0.3559783   2.7364130   6.8179348
## 2004  -9.7853261   1.3559783   1.6114130   5.8179348
## 2005  -8.6603261   2.7309783   0.3614130   5.4429348
## 2006  -7.6603261   0.2309783   1.4864130   5.9429348
## 2007  -7.6603261   0.2309783   1.4864130   5.8179348
## 2008  -8.1603261   0.9809783   4.6114130   4.6929348
## 2009 -10.2853261   0.3559783   0.9864130   6.5679348
## 2010  -7.6603261   0.9809783   1.4864130   6.3179348
## 2011  -9.1603261   1.2309783   3.4864130   4.6929348
## 2012 -10.2853261   1.8559783   2.2364130   6.6929348
## 2013 -12.0353261   4.1059783   1.7364130   7.5679348
## 2014 -10.1603261  -2.6440217          NA          NA
## 
## $figure
## [1]  5.2853261  0.1440217 -0.3614130 -5.0679348
## 
## $type
## [1] "additive"
## 
## attr(,"class")
## [1] "decomposed.ts"
\end{verbatim}

\begin{Shaded}
\begin{Highlighting}[]
\KeywordTok{plot}\NormalTok{(PIB.decomposto)}
\end{Highlighting}
\end{Shaded}

\includegraphics{Trabalho_I_files/figure-latex/unnamed-chunk-26-1.pdf}

\begin{enumerate}
\def\labelenumi{\alph{enumi})}
\setcounter{enumi}{2}
\tightlist
\item
  Usando o índice dos últimos 12 anos, encontre uma projeção para o
  PIB(índice) do próximo semestre usando um modelo AR(1). Neste caso use
  a série das diferenças.
\end{enumerate}

Usando Predict:

\begin{Shaded}
\begin{Highlighting}[]
\ControlFlowTok{if}\NormalTok{(}\OperatorTok{!}\KeywordTok{require}\NormalTok{(forecast)) \{}
  \KeywordTok{install.packages}\NormalTok{(}\StringTok{"forecast"}\NormalTok{)}
  \KeywordTok{library}\NormalTok{(forecast)}
\NormalTok{\}}
\end{Highlighting}
\end{Shaded}

\begin{verbatim}
## Loading required package: forecast
\end{verbatim}

\begin{Shaded}
\begin{Highlighting}[]
\NormalTok{PIB.dif <-}\StringTok{ }\KeywordTok{diff}\NormalTok{(PIB.ts[}\DecValTok{49}\OperatorTok{:}\DecValTok{96}\NormalTok{])}
\NormalTok{PIB.predict <-}\StringTok{ }\KeywordTok{predict}\NormalTok{(}\KeywordTok{auto.arima}\NormalTok{(PIB.dif),}\DataTypeTok{ahead =} \DecValTok{1}\NormalTok{)}
\NormalTok{PIB.predict}
\end{Highlighting}
\end{Shaded}

\begin{verbatim}
## $pred
## Time Series:
## Start = 48 
## End = 48 
## Frequency = 1 
## [1] -6.898466
## 
## $se
## Time Series:
## Start = 48 
## End = 48 
## Frequency = 1 
## [1] 5.194559
\end{verbatim}

Usando Forecast:

\begin{Shaded}
\begin{Highlighting}[]
\NormalTok{etsfit.PIB.dif <-}\StringTok{ }\KeywordTok{ets}\NormalTok{(PIB.dif)}
\NormalTok{etsfit.PIB.dif}
\end{Highlighting}
\end{Shaded}

\begin{verbatim}
## ETS(A,N,N) 
## 
## Call:
##  ets(y = PIB.dif) 
## 
##   Smoothing parameters:
##     alpha = 1e-04 
## 
##   Initial states:
##     l = 1.5736 
## 
##   sigma:  5.4275
## 
##      AIC     AICc      BIC 
## 343.9129 344.4711 349.4634
\end{verbatim}

\begin{Shaded}
\begin{Highlighting}[]
\KeywordTok{accuracy}\NormalTok{(etsfit.PIB.dif)}
\end{Highlighting}
\end{Shaded}

\begin{verbatim}
##                       ME     RMSE      MAE      MPE    MAPE      MASE
## Training set 0.001407463 5.310812 3.681257 59.63687 88.9061 0.6271772
##                    ACF1
## Training set -0.2199867
\end{verbatim}

\begin{Shaded}
\begin{Highlighting}[]
\NormalTok{fcast.PIB.dif <-}\StringTok{ }\KeywordTok{forecast}\NormalTok{(etsfit.PIB.dif)}
\NormalTok{fcast.PIB.dif}
\end{Highlighting}
\end{Shaded}

\begin{verbatim}
##    Point Forecast     Lo 80    Hi 80     Lo 95    Hi 95
## 48        1.57364 -5.382041 8.529321 -9.064157 12.21144
## 49        1.57364 -5.382041 8.529321 -9.064157 12.21144
## 50        1.57364 -5.382041 8.529321 -9.064157 12.21144
## 51        1.57364 -5.382042 8.529321 -9.064157 12.21144
## 52        1.57364 -5.382042 8.529322 -9.064157 12.21144
## 53        1.57364 -5.382042 8.529322 -9.064157 12.21144
## 54        1.57364 -5.382042 8.529322 -9.064157 12.21144
## 55        1.57364 -5.382042 8.529322 -9.064157 12.21144
## 56        1.57364 -5.382042 8.529322 -9.064157 12.21144
## 57        1.57364 -5.382042 8.529322 -9.064157 12.21144
\end{verbatim}

\begin{Shaded}
\begin{Highlighting}[]
\KeywordTok{plot}\NormalTok{(fcast.PIB.dif)}
\end{Highlighting}
\end{Shaded}

\includegraphics{Trabalho_I_files/figure-latex/unnamed-chunk-28-1.pdf}

Usando ARIMA:

\begin{Shaded}
\begin{Highlighting}[]
\NormalTok{arimafit.PIB.dif <-}\StringTok{ }\KeywordTok{auto.arima}\NormalTok{(PIB.dif)}
\NormalTok{fcast.ARIMA.PIB.dif <-}\StringTok{ }\KeywordTok{forecast}\NormalTok{(arimafit.PIB.dif)}
\NormalTok{fcast.ARIMA.PIB.dif}
\end{Highlighting}
\end{Shaded}

\begin{verbatim}
##    Point Forecast      Lo 80      Hi 80      Lo 95     Hi 95
## 48     -6.8984662 -13.555561 -0.2413712 -17.079614  3.282682
## 49      4.4612251  -2.631513 11.5539636  -6.386182 15.308633
## 50      0.2848823  -6.864703  7.4344677 -10.649465 11.219230
## 51      1.8202972  -5.336937  8.9775316  -9.125748 12.766343
## 52      1.2558084  -5.902459  8.4140760  -9.691817 12.203434
## 53      1.4633403  -5.695067  8.6217476  -9.484499 12.411179
## 54      1.3870421  -5.771384  8.5454682  -9.560826 12.334910
## 55      1.4150928  -5.743336  8.5735215  -9.532779 12.362965
## 56      1.4047801  -5.753649  8.5632091  -9.543092 12.352653
## 57      1.4085715  -5.749858  8.5670006  -9.539301 12.356444
\end{verbatim}

\begin{Shaded}
\begin{Highlighting}[]
\KeywordTok{plot}\NormalTok{(fcast.ARIMA.PIB.dif)}
\end{Highlighting}
\end{Shaded}

\includegraphics{Trabalho_I_files/figure-latex/unnamed-chunk-29-1.pdf}

Usando Forecast (sem usar a diferença):

\begin{Shaded}
\begin{Highlighting}[]
\NormalTok{etsfit.PIB.ts <-}\StringTok{ }\KeywordTok{ets}\NormalTok{(PIB.ts[}\DecValTok{49}\OperatorTok{:}\DecValTok{96}\NormalTok{])}
\NormalTok{etsfit.PIB.ts}
\end{Highlighting}
\end{Shaded}

\begin{verbatim}
## ETS(M,A,N) 
## 
## Call:
##  ets(y = PIB.ts[49:96]) 
## 
##   Smoothing parameters:
##     alpha = 0.0296 
##     beta  = 0.0296 
## 
##   Initial states:
##     l = 25.1972 
##     b = 1.3539 
## 
##   sigma:  0.0825
## 
##      AIC     AICc      BIC 
## 326.5554 327.9840 335.9114
\end{verbatim}

\begin{Shaded}
\begin{Highlighting}[]
\KeywordTok{accuracy}\NormalTok{(etsfit.PIB.ts)}
\end{Highlighting}
\end{Shaded}

\begin{verbatim}
##                      ME     RMSE      MAE       MPE     MAPE      MASE
## Training set 0.01216636 4.313633 2.950698 -1.043802 5.960771 0.7456065
##                     ACF1
## Training set -0.04256709
\end{verbatim}

\begin{Shaded}
\begin{Highlighting}[]
\NormalTok{fcast.PIB.ts <-}\StringTok{ }\KeywordTok{forecast}\NormalTok{(etsfit.PIB.ts)}
\NormalTok{fcast.PIB.ts}
\end{Highlighting}
\end{Shaded}

\begin{verbatim}
##    Point Forecast    Lo 80    Hi 80    Lo 95     Hi 95
## 49       75.54042 67.55617 83.52466 63.32957  87.75127
## 50       76.91158 68.76863 85.05453 64.45801  89.36515
## 51       78.28274 69.96423 86.60126 65.56067  91.00482
## 52       79.65391 71.13682 88.17100 66.62814  92.67967
## 53       81.02507 72.28071 89.76944 67.65172  94.39842
## 54       82.39623 73.39078 91.40169 68.62358  96.16889
## 55       83.76740 74.46259 93.07221 69.53692  97.99787
## 56       85.13856 75.49243 94.78469 70.38608  99.89105
## 57       86.50972 76.47736 96.54209 71.16654 101.85291
## 58       87.88089 77.41520 98.34658 71.87500 103.88678
\end{verbatim}

\begin{Shaded}
\begin{Highlighting}[]
\KeywordTok{plot}\NormalTok{(fcast.PIB.ts)}
\end{Highlighting}
\end{Shaded}

\includegraphics{Trabalho_I_files/figure-latex/unnamed-chunk-30-1.pdf}

Usando ARIMA (sem usar a diferença):

\begin{Shaded}
\begin{Highlighting}[]
\NormalTok{arimafit.PIB.ts <-}\StringTok{ }\KeywordTok{auto.arima}\NormalTok{(PIB.ts[}\DecValTok{49}\OperatorTok{:}\DecValTok{96}\NormalTok{])}
\NormalTok{fcast.ARIMA.PIB.ts <-}\StringTok{ }\KeywordTok{forecast}\NormalTok{(arimafit.PIB.ts)}
\NormalTok{fcast.ARIMA.PIB.ts}
\end{Highlighting}
\end{Shaded}

\begin{verbatim}
##    Point Forecast    Lo 80     Hi 80    Lo 95     Hi 95
## 49       87.10154 80.44444  93.75863 76.92039  97.28269
## 50       91.56276 83.68634  99.43918 79.51681 103.60871
## 51       91.84764 82.45912 101.23617 77.48913 106.20615
## 52       93.66794 83.13325 104.20263 77.55652 109.77936
## 53       94.92375 83.30514 106.54236 77.15461 112.69289
## 54       96.38709 83.79469 108.97949 77.12868 115.64550
## 55       97.77413 84.27211 111.27616 77.12457 118.42370
## 56       99.18923 84.83715 113.54130 77.23963 121.13883
## 57      100.59401 85.43878 115.74923 77.41609 123.77193
## 58      102.00258 86.08492 117.92024 77.65862 126.34654
\end{verbatim}

\begin{Shaded}
\begin{Highlighting}[]
\KeywordTok{plot}\NormalTok{(fcast.ARIMA.PIB.ts)}
\end{Highlighting}
\end{Shaded}

\includegraphics{Trabalho_I_files/figure-latex/unnamed-chunk-31-1.pdf}


\end{document}
